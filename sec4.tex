\documentclass[./main]{subfiles}
\begin{document}
\section{Reduction to the case of Hecke \texorpdfstring{$\L$}{L}-functions}\setlength{\headheight}{22.54279pt}

Our definition (\ref{(3.5)}) of $F_C(s)$ was in terms of Artin $\L$-functions corresponding to the (usually nonlinear) characters of $G(L/K)$. In this section we show that $F_C(s)$ can be written in terms of Hecke (abelian) $\L$-functions. This will enable us to obtain much better results on the location and density of the singularities of $F_C(s)$. The reduction we will use is due to Deuring \cite{3-Deuring1935berDT} (later rediscovered by MacCluer \cite{9-maccluer1968}). We learned of it from \cite{10-Moreno}, and should like to thank J.~P. Serre for bringing Moreno's paper to our attention and for supplying the following formulation of Deuring's idea.

In defining $F_C(s)$ by (\ref{(3.5)}), we have already selected an element $g \in C$. Let $H = \langle g\rangle$ be the cyclic group generated by $g$, $E$ the fixed field of $H$, and let $\chi$ denote the irreducible characters of $H$. Since $H$ is cyclic, the characters $\chi$ are one-dimensional. We will retain this notation for the rest of this paper.
\begin{lemma} \label{4.1}
We have
\[\tag{4.1}\label{(4.1)}
    F_C(s) = -\frac{\small|C\small|}{\small|G\small|} \, \sum_{\chi}\,{\xoverline{\chi}(g)\,\frac{\L'}{\L}\,(s,\chi,L/E)}.
\]
\end{lemma}
\begin{proof}
Let $\tau : H \longrightarrow \C$ be the class function defined by
\[
\tau(h) = 
\begin{cases} 
\small|H\small| & \text{ if } h =g, \\
0 & \text{ if } h \neq g.
\end{cases}
\]
Then the orthogonality relations for characters of $H$ imply that
\[ \tau = \sum_{\chi}\:{\xoverline{\chi}(g)\,\chi}.\]
Let $\tau^*$ denote the class function on $G$ induced by $\tau$, which by direct calculation equals
\[
\tau^*(y) = 
\begin{cases} 
\big|C_G(g)\big| &  y\in C, \\
0 & y\notin C,
\end{cases}
\]
where $C_G(g)$ is the centralizer of $g$ in $G$. Now $\big|C_G(g)\big|\big|C\big| = \big|G\big|$ so that $\tau^* = f_C$ [see (\ref{(3.4)})]. This implies
\[\sum_{\chi}{\xoverline{\chi}(g)\chi^*} = \sum_{\phi}{\xoverline{\phi}(g)\phi}, \]
so that for $\Re{(s)}>1$ we have
\[\tag{4.2}\label{4.2}
    F_C(s) = -\frac{\small|C\small|}{\small|G\small|} \: \sum_{\chi}\:{\xoverline{\chi}(g)\,\frac{\L'}{\L}\,(s,\chi^*,L/K)}.
\]
But $\L(s,\chi^*,L/K) = \L(s,\chi,L/E)$, and so (\ref{4.1}) holds for $\Re{(s)}>1$, and therefore (by analytic continuation) for all $s$.
\end{proof}
\end{document}
%Notes:
% spacing in lemma 4.1 and 4.2 equations, spread the symbols out a bit.
%   What kind of arrow to use for the mapsto arrow? I've used \to for now.
%   What is the deal with 3.5? Is it a lemma or equation? parentheses around its reference or nah?
% center aligning the terms in the cases environment --- idk how to do that.
%I (Marcmelwin) am done writing this, feel free to edit