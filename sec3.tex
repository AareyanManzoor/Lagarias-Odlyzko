\documentclass[./main]{subfiles}
\begin{document}
\section{Artin \texorpdfstring{$L$}{L}-functions and Mellin transform}

In this section we establish the relation between $\psi_C(x)$ and a certain truncated inverse Mellin transform. Throughout this and subsequent sections we will use the abbreviations $\pi_C(x)$, $\psi_C(x)$ and $N$ for $\pi_C(x, L/K)$, $\psi_C(x, L/K)$, and $N_{K/Q}$, respectively. We will also use $\phi$ to denote irreducible characters of $G=G(L/K)$.

For each irreducible character $\phi$ of $G$ we define
\[\tag{3.1}\label{3.1}
\phi_K(P^m)=\frac{1}{e}\sum_{\alpha \in I} \phi(\tau^m\alpha),\]
where $I$ is the inertia group of $P$, one of the prime ideal factors of $\mathcal P$, $e=|I|$ and $\tau$ is one of the Frobenius automorphisms corresponding to $\mathcal P$. If $L(s, \phi, L/K)$ is the Artin L-series associated to $\phi$, then for $\text{Re}(s)>1$ we have
\[\tag{3.2}\label{3.2}
-\frac{L'}{L}(s, \phi, L/K)=\sum_\mathcal{P} \sum_{m=1}^\infty \phi_K(\mathcal{P}^m)\log (N \mathcal{P})(N\mathcal{P})^{-ms}\]
where the outer sum is over all the prime ideals of $K$. We should also note that the definitions (\ref{3.1}) and (\ref{3.2}) apply equally well to reducible characters.

To single out those $\mathcal{P}^m$ with $\ArtinSymbol{L/K}{\mathcal{P}}^m=C$, we will use the characters $\phi$.
\end{document}