\documentclass[./main]{subfiles}
\begin{document}
\section{Introduction}

Let $K$ be an algebraic number field (finite extension of the rationals $\mathbb Q$) and $L$ a normal extension of $K$ with Galois group $G = G(L/K)$. Let $\disc_L$ and $\disc_K$ denote the absolute values of the discriminants of $L$ and $K$, respectively, and let $n_L = [L:Q]$, $n_K = [K:Q]$. Throughout this paper $\p$ will denote a prime ideal of $K$ and $P$ a prime ideal of $L$. If $\p$ is a prime ideal of $K$ which is unramified in $L$, then we use the Artin symbol $\ArtinSymbol{L/K}{\mathfrak p}$ to denote the conjugacy class of Frobenius automorphisms corresponding to prime ideals $P\mid \p$. For each conjugacy class $C$ of $G$, we define
\[ \pi_C (x, L/K) = \Bigg\vert \Big\{\p : \p \text{ unramified in } L,\, \ArtinSymbol{L/K}{\p} = C,\, \norm_{K/Q}\p \le x \Big\} \Bigg\vert. \]

The Chebotarev density theorem \cite{15-Tschebotareff1926} asserts that
\[\tag{1.1}\label{(1.1)} \pi_C (x, L/K) \sim \frac{|C|}{|G|} \Li(x) \hspace{1em} \text{as } x\to\infty, \]

where $\Li(x)$ is the familiar logarithmic integral
\[ \Li(x) = \int_2^x \frac{\dd t}{\log t} \sim \frac{x}{\log x} \hspace{1em}\text{as }x\to\infty \]

The Chebotarev density theorem generalizes many of the classical results on the distribution of primes and prime ideals. For example, if we consider the trivial extension $L = K$ of $K$ ($K$ does not have to be normal over $\mathbb Q$, then there is only one conjugacy class, and (\ref{(1.1)}) shows that the number of prime ideals of $K$ with norm $\le x$ is asymptotic to $\Li(x)$, which is exactly the prime ideal theorem. If we let $K = \mathbb Q$ and $L = \mathbb Q(e^{2\pi i/q})$, then the conjugacy classes of $G$ correspond to the residue classes modulo $q$, and (\ref{(1.1)}) gives us the prime number theorem for arithmetic progressions.

One of the most important of the many applications of the Chebotarev density theorem deals with the group of an equation. Suppose that $f(x)$ is a monic polynomial whose coefficients are algebraic integers in $K$ and which is irreducible over $K$. Suppose further that $L$ is the splitting field of $f(x)$ over $K$. If we regard $G = G(L/K)$ as a permutation group acting on the roots of $f(x)$, then for almost all prime ideals $\p$ of $K$ the cycle structure of $\ArtinSymbol{L/K}{\p}$ depends on the factorization of $f(x)$ modulo $\p$, and vice versa. Thus if $G$ is known, then the Chebotarev density theorem tells us how often various factorizations occur as $\p$ runs through all the prime ideals of $K$. On the other hand, if we do not know $G$, then factoring $f(x)$ modulo the prime ideals of $K$ will yield the complete cycle structures of $G$, since by (\ref{(1.1)}) for every conjugacy class $C$ there are infinitely many primes $\p$ with $\ArtinSymbol{L/K}{\p} = C$. This can be very helpful in the determination of $G$ [\cite{16-waerden1970}, vol. 1, pp. 189-192], especially since by considering enough primes we can even determine the relative densities of elements of $G$ which have a given  cycle structure. (Unfortunately, sometimes this is not enough to determine $G$ completely, since it is possible to construct two nonisomorphic groups which have transitive permutation representations in which the number of elements with a given cycle structure is the same for both groups.) In these situations it is important to be able to compute a bound below which every conjugacy class will occur as the Artin symbol of a prime ideal of $K$.

The usual proofs of the Chebotarev theorem contains either no error estimates at all, or else estimates which contain constants depending in some undetermined way on the fields $K$ and $L$. In particular, such estimates do not allow us to specify effectively a value $x_0 = x_0(L/K)$ such that
% \[ \tag{


\end{document}