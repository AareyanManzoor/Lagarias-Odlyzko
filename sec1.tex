\documentclass[./main]{subfiles}
\begin{document}
\section{Introduction}

Let $K$ be an algebraic number field (finite extension of the rationals $\mathbb Q$) and $L$ a normal extension of $K$ with Galois group $G = G(L/K)$. Let $\disc_L$ and $\disc_K$ denote the absolute values of the discriminants of $L$ and $K$, respectively, and let $n_L = [L:\Q]$, $n_K = [K:\Q]$. Throughout this paper $\p$ will denote a prime ideal of $K$ and $P$ a prime ideal of $L$. If $\p$ is a prime ideal of $K$ which is unramified in $L$, then we use the Artin symbol $\ArtinSymbol{L/K}{\mathfrak p}$ to denote the conjugacy class of Frobenius automorphisms corresponding to prime ideals $P\mid \p$. For each conjugacy class $C$ of $G$, we define
\[ \pi_C (x, L/K) = \bigg\vert \Big\{\p : \p \text{ unramified in } L,\, \ArtinSymbol{L/K}{\p} = C,\, \norm_{K/\Q}\p \le x \Big\} \bigg\vert. \]
The Chebotarev density theorem \cite{15-Tschebotareff1926} asserts that
\[\tag{1.1}\label{(1.1)} \pi_C (x, L/K) \sim \frac{|C|}{|G|} \Li(x) \hspace{1em} \text{as } x\to\infty, \]
where $\Li(x)$ is the familiar logarithmic integral
\[ \Li(x) = \int_2^x \frac{\dd t}{\log t} \sim \frac{x}{\log x} \hspace{1em}\text{as }x\to\infty \]

The Chebotarev density theorem generalizes many of the classical results on the distribution of primes and prime ideals. For example, if we consider the trivial extension $L = K$ of $K$ ($K$ does not have to be normal over $\mathbb Q$), then there is only one conjugacy class, and (\ref{(1.1)}) shows that the number of prime ideals of $K$ with norm $\le x$ is asymptotic to $\Li(x)$, which is exactly the prime ideal theorem. If we let $K = \mathbb Q$ and $L = \mathbb Q(e^{2\pi i/q})$, then the conjugacy classes of $G$ correspond to the residue classes modulo $q$, and (\ref{(1.1)}) gives us the prime number theorem for arithmetic progressions.

One of the most important of the many applications of the Chebotarev density theorem deals with the group of an equation. Suppose that $f(x)$ is a monic polynomial whose coefficients are algebraic integers in $K$ and which is irreducible over $K$. Suppose further that $L$ is the splitting field of $f(x)$ over $K$. If we regard $G = G(L/K)$ as a permutation group acting on the roots of $f(x)$, then for almost all prime ideals $\p$ of $K$ the cycle structure of $\ArtinSymbol{L/K}{\p}$ depends on the factorization of $f(x)$ modulo $\p$, and vice versa. Thus if $G$ is known, then the Chebotarev density theorem tells us how often various factorizations occur as $\p$ runs through all the prime ideals of $K$. On the other hand, if we do not know $G$, then factoring $f(x)$ modulo the prime ideals of $K$ will yield the complete cycle structures of $G$, since by (\ref{(1.1)}) for every conjugacy class $C$ there are infinitely many primes $\p$ with $\ArtinSymbol{L/K}{\p} = C$. This can be very helpful in the determination of $G$ \cite[vol. 1, pp. 189-192]{16-waerden1970}, especially since by considering enough primes we can even determine the relative densities of elements of $G$ which have a given  cycle structure. (Unfortunately, sometimes this is not enough to determine $G$ completely, since it is possible to construct two nonisomorphic groups which have transitive permutation representations in which the number of elements with a given cycle structure is the same for both groups.) In these situations it is important to be able to compute a bound below which every conjugacy class will occur as the Artin symbol of a prime ideal of $K$.

The usual proofs of the Chebotarev theorem contains either no error estimates at all, or else estimates which contain constants depending in some undetermined way on the fields $K$ and $L$. In particular, such estimates do not allow us to specify effectively a value $x_0 = x_0(L/K)$ such that
\[ \pi_C(x, L/K) > 0 \hspace{1em}\text{if} \hspace{1em} x\ge x_0. \tag{1.2}\label{(1.2)} \]

The purpose of this paper is to prove two versions of the Chebotarev theorem, each of which has an error term which is an explicit and effectively computable function of $x$, $n_L$, $\disc_L$, and $|C|/|G|$. One version assumes the truth of the Generalized Riemann Hypothesis (GRH) and the other holds unconditionally.

We first state the conditional result.

\begin{theorem} \label{1.1}
There exists an effectively computable positive absolute constant $c_1$ such that if GRH holds for the Dedekind zeta function of $L$, then for every $x>2$, 
\[ \bigg| \pi_C(x, L/K) - \frac{|C|}{|G|} \Li(x) \bigg| \le c_1 \bigg\{ \frac{|C|}{|G|} x^{\frac{1}{2}} \log(\disc_L x^{n_L}) + \log \disc_L \bigg\}.\tag{1.3} \label{(1.3)} \]
\end{theorem}

This theorem yields immediately a value of $x_0$ such that (\ref{(1.2)}) holds. (We utilize here the estimate $n_L^{-1} \log \disc_L > 1 + \eps$ for some $\eps > 0$, valid for $n_L > 1$. It follows from Minkowski's discriminant bound, and it can also be derived from (\ref{(5.11)}) (see \cite{11-odlyzko1977}). 

\begin{corollary} \label{1.2}
There exists an effectively computable positive absolute constant $c_2$ such that if the GRH holds for the Dedekind zeta function of $L \ne \mathbb Q$, then for every conjugacy class of $G$ there exists an unramified prime ideal $\p$ in $K$ such that $\ArtinSymbol{L/K}{\p} = C$ and 
\[ \tag{1.4}\label{(1.4)} \norm_{K/\mathbb Q} \p \le c_2(\log \disc_L)^2 (\log \log \disc_L)^4. \]
(If $L = \mathbb Q$, $\p = (2)$ yields a solution.)
\end{corollary}


At the end of this paper we will indicate how the above estimate can be improved so as to eliminate the $\log \log \disc_L$ term.

We next state the unconditional result.

\begin{theorem} \label{1.3}
If $n_L > 1$ then $\zeta_L(s)$ has at most one zero in the region defined by $s = \sigma +it$  with 
\[ \tag{1.5}\label{(1.5)} 1 - (4\log\disc_L)^{-1} \le \sigma \le 1, \hspace{1em} |t| \le (4\log\disc_L)^{-1}. \]
(If $n_L = 1$, $L = \mathbb Q$ and there is no zero in $|t| \le 14$, $\sigma > 0$.)
\newline If such a zero exists, it must be real and simple, and we denote it by $\beta_0$. 

Further, there exist absolute effectively computable constant $c_3$ and $c_4$ such that if
\[ \tag{1.6}\label{(1.6)} x\ge \exp (10 n_L (\log \disc_L)^2 ), \]
then
\[ \tag{1.7}\label{(1.7)}
    \bigg| \pi_C (x) - \frac{|C|}{|G|} \Li(x) \bigg| \le \frac{|C|}{|G|} \Li (x^{\beta_0}) + c_3x \exp \big(-c_4n_L^{-\frac{1}{2}}(\log x)^{\frac 12}\big),
\]
where the $\beta_0$ term is present only when $\beta_0$ exists.
\end{theorem}

Because of the presence of the $\beta_0$ factor, Theorem \ref{1.3} does not fully meet our criterion of effectiveness, which is that the error term should depend only on $x$, $n_L$, $\disc_L$, and $|C|/|G|$. However, this defect can be remedied by utilizing any effective bound for $\beta_0$. In most cases the best known such bound is that of Stark \cite[p.148]{13-Stark1974} which we quote below.

\begin{theorem}\label{1.4}
Let the notation be as in Theorem \ref{1.3}, and let $m_L = 4$ if $1$ is normal over $\mathbb Q$, $m_L = 16$ if there is a sequence of fields \[\mathbb Q = k_1 \subset k_1 \subset \cdots \subset k_r = L\] with each field normal over the preceding one, and $m_L = 4n_L!$ otherwise. Then there exists an effectively computable absolute constant $c_5$ such that
\[ \beta_0 < \max \bigl[1- (m_L\log\disc_L)^{-1},\, 1-(c_5\disc_L^{1/n_L})^{-1}\bigr]. \tag{1.8}\label{(1.8)} \]
\end{theorem}

Even if $\beta_0$ does not exist, Theorem \ref{1.3} does not give a good unconditional bound for the smallest norm of a prime ideal whose Artin symbol is a given conjugacy class. A reasonable conjecture might be that there should be an effectively computable absolute constant $c$ such that for every normal extension $L/K$ and every conjugacy class $C$ of $G(L/K)$, there should be an unramified $\p$ with $\ArtinSymbol{L/K}{\p} = C$ and 
\[ \norm_{K/\mathbb Q}\p \le (\log \disc_L)^c. \tag{1.9}\label{(1.9)} \]
When $L$ is a cyclomic extension of $K = \mathbb Q$, (\ref{(1.9)}) is equivalent to Linnik's theorem \cite[p.39]{1-Bombieri1974}. However, if $K = \mathbb Q$ and $L = \mathbb Q(\sqrt{d})$ is a quadratic extension of $\mathbb Q$, the determinantion of the least prime $p$ with $\ArtinSymbol{L/\mathbb Q}{(p)} \ne \{1\}$ corresponds to the problem of determining the least quadratic nonresidue (mod $d$), and for this problem no unconditional bound better than
\[ p \le c_6 \disc_L^{c_7} \tag{1.10}\label{(1.10)} \]
is known, where $c_6$ and $c_7$ are positive constants. Thus without some major new ideas it would probably be very difficult to prove an unconditional result as good as (\ref{(1.9)}). However, by using slightly different techniques (which are designed to detect prime ideals rather than estimate their total number) one can prove the following result \cite{7-lagarias1979}.

\begin{theorem*}
There exist effectively computable positive absolute constants $b_1$ and $b_24$ such that for every conjugacy class $C$ of $G$ there exists an unramified prime ideal $\p$ of $K$ such that $\ArtinSymbol{L/K}{\p} = C$ and
\[ \norm_{K/\Q} \p \le b_1 \disc_L^{b_2}. \]
\end{theorem*}

The approach used in this paper has a long history. The argument given here may be viewed as a direct descendent of de la Vallee Poussin's proof of the prime number theorem. We follow closely with the pattern of Davenport's treatment \cite{2-davenport2013multiplicative} of the prime number theorem for arithmetic progressions. The main innovation here is the careful treatment of the dependencies of various constants on $n_L$ and $\disc_L$ (cf. \cite{2-davenport2013multiplicative, 4.1-Fogels1961, 5-goldstein1970, 8-lang1971, 10-Moreno}).

Aside from some slight acquaintance with algebraic and analytic number theory, this paper also assumes knowledge of the basic properties of Hecke and Artin $L$-functions \cite{6-Hilbronn1967}. The deepest of these results is the abelian reciprocity law, which tells us that an abelian Artin $L$-series is a Hecke $L$-series, and so is analytic for $s\ne 1$. 

Throughout this paper $c_1, c_2, \dots$ will denote effectively computable positive absolute constants. (In particular, they are independent of $K$ and $L$.) The Vinogradov notation
\[ f \Vinogradov g \]
will be used to denote the existence of an effectively computable positive absolute constant $A$ (not necessary the esame in each occurance) such that
\[ |f| \le Ag, \]
in the range indicated.

\end{document}