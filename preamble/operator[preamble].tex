\newcommand{\plscite}[1]{{\color{purple} #1}} 

\newcommand{\ArtinSymbol}[2]{\bigg[\dfrac{#1}{#2}\bigg]} %\ArtinSymbol{L/K}{\mathfrak{p}} for artin symbol [(L/K)/p] for example
\DeclareMathOperator{\Li}{Li} %logarithmic integral
\newcommand{\disc}{\Delta} %for the discriminant, in book this is d_L, d_K. Here it shall be \disc_L, \disc_K
\DeclareMathOperator{\norm}{N} %for the galois norm, \norm_{L/K} (\p) for example.
\newcommand{\Vinogradov}{\ll} % see page 417, its called the Vinogradov notation there.
\newcommand{\dd}{\mathrm{d}}% differential in e.g. integral, \int \dd x


\newcommand{\defemph}[1]{\textbf{#1}}

%%%% things to use instead of tilde, hat, overline,bar etc, for good size. 
\newcommand{\xoverline}[1]{\mskip.5\thinmuskip\overline{\mskip-.5\thinmuskip {#1} \mskip-.5\thinmuskip}\mskip.5\thinmuskip}
\newcommand{\xhat}[1]{\mskip.5\thinmuskip\widehat{\mskip-.5\thinmuskip {#1} \mskip-.5\thinmuskip}\mskip.5\thinmuskip}
\newcommand{\xtilde}[1]{\mskip.5\thinmuskip\widetilde{\mskip-.5\thinmuskip {#1} \mskip-.5\thinmuskip}\mskip.5\thinmuskip}

%%%%% shortcuts
\newcommand{\p}{\mathfrak{p}}
\newcommand{\Q}{\mathbb{Q}}
\newcommand{\R}{\mathbb{R}}
\newcommand{\C}{\mathbb{C}}
