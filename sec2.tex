\documentclass[./main]{subfiles}
\begin{document}
\section{Outline of the main argument}

The main argument is primarily concerned with the derivation of an asymptotic formula with an explicit error term ofr a weighted prime-power-counting functiong $\phi_C(x) = \phi_C(x, L/K)$ associated to $\pi_C(x, L/K)$.
It is defined by

\[
\phi_C(x, L/K) = \sum_{\substack{N_{K/\Q}  \p^m \leq x \\ \p\text{ unramified} \\ \ArtinSymbol{L/K}{\p}^m = C }} \log(N_{K/Q} \p)
\]

The details of this argument are complicated, the main steps are simple in conception:

\begin{enumerate}[(i)]
    \item $\phi_C(x)$ differs from a truncate inverse Mellin transform
    
    \[
    I_C(x, T) = \frac{1}{2\pi i} \int_{\sigma_0 - iT}^{\sigma_0 + iT} F_C(s) \frac{x^s}{s} \dd s,
    \]
    
    by a remainder term $R_1(x, T)$.
    
    \item $F_C(s)$ can in fact be written as a linear combination of logarithmic derivatives of Hecke (abelian) L-functions.
    As a consequence, all the singularities of $F_C(s)$, which are simply poles only, occur at zeroes and pole of $\zeta_L(s)$.
    
    \item $I_C(x, T)$ differs from a certain contour integral
    
    \[
    B_C(x, T) = \frac{1}{2\pi i} \oint_{B_T} F_C(s) \frac{x^s}{s} \dd s
    \]
    by a remainder term $R_2(x, T)$.
    This step is traditionally labelled ``shifting the line of integration to the left.''
    
    Certain results on the density of zeros of $\zeta_L(s)$ in the critical strip $0 < \Re s < 1$ are necessary to estimate $R_2(x, T$.
    
    \item The contour integral $B_C(x, T)$ is evaluated by Cauchy's residue theorem.
    The integrand has poles at the zeroes and the pole of $\zeta_L(s)$, and the result is a main term $\frac{|C|}{|G|}x$ coming from the pole of $\zeta_L(s)$ at $s = 1$, together with a certain sum $S(x, T)$ over the zeroes of $\zeta_L(s)$ within the contour $B_T$.
    
    The end result of these steps is a truncated ``explicit formula'' for $\phi_C(x)$ with an unconditional error term, which is stated as Theorem 7.1.
    
    \item The sum over the zeroes $S(x, T)$ is estimated.
    It is at this point that unproved hypothesis about the zeroes can be helpful.
    An unconditional upper bound for $|S(x, T)|$ is obtained using the existence of a zero-free region of $\zeta_L(s)$ near the vertical line $\sigma =1$.
    A much better estimate for $|S(x, T)|$ is made assuming the Generalized Riemann hypothesis for $\zeta_L(s)$.
    
    \item The asymptotic formula $\phi_C(x) \sim \frac{|C|}{|G|}x$ with an explicit remainder term is derived by making an appropriate choice of $T$ as a function of $x$, to minimize the accumulated error terms.
    (This choice depends on whether the GRH is assumed or not, of course.)
    
    \item The asymptotic formula $\pi_C(x) \sim \frac{|C|}{|G|}\Li(x)$ with an explicit remainder term is derived by partial summation from that for $\phi_C(x)$.
    
    The remaining sections of this paper carry out the details (although we will not follow this outline exactly).
\end{enumerate}
\end{document}