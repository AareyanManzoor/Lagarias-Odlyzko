\documentclass[./main]{subfiles}
\begin{document}
\section{Final estimates}

We conclude this paper by applying the explicit formula of Theorem \ref{07-THM-7.1} to estimate $\psi_C(x)$ and $\pi_C(x)$.
We start with the GRH estimate for $\psi_C(x)$, which is the easiest to obtain.

\begin{theorem}\label{09-THM-9.1}
If $\zeta_L(s)$ satisfies the GRH, then 
\[\tag{9.1}\label{(9.1)}
\psi_C(x) - \frac{|C|}{|G|} x \ll \frac{|C|}{|G|} x^{\frac{1}{2}} \log x \log \disc_L x^{n_L} + \log x \log \disc_L 
\]
for all $x \geq 2$.
\end{theorem}
\begin{proof}
If $\zeta_L(s)$ satisfies the GRH, then so do all of the $\L(s, \chi)$.
Therefore, for each $\chi$ there are no nontrivial zeros $\rho$ with $|\rho| < 1/2$, and so by Lemma \ref{5.4}.

\begin{align*}
    \Bigg| \sum_{\substack{\rho \\ |\gamma| < T}} \frac{x^\rho}{\rho} + \sum_{|\rho| < \frac{1}{2}} \frac{1}{\rho} \Bigg| & \leq x^{\frac{1}{2}} \sum_{\substack{\rho \\ |\gamma| < T}} \frac{1}{|\rho|} \\
    & \ll x^{\frac{1}{2}} \sum_{j=1}^{\lfloor T\rfloor} \frac{n_X(j)}{j} \\
    & \ll x^{\frac{1}{2}} (\log A(\chi) + n_E \log T)\log T,
\end{align*}
which together with (\ref{(7.5)}) implies 
\[\tag{9.2}\label{(9.2)}
S(x, T) \ll \frac{|C|}{|G|} x^{\frac{1}{2}} (\log \disc_L  + n_L \log T) \log T
\]
for all $T \geq 2$.
We choose $T = x^\frac{1}{2} + 1$, way, and then (\ref{(9.2)}) and (\ref{(7.4)}) imply (\ref{(9.1)}) for $x \geq 2$.
\end{proof}

\begin{theorem}\label{09-THM-9.2}
There is an effectively computable positive absolute constant $c_{13}$ such that if 
\[\tag{9.3}\label{(9.3)}
x \geq \exp(4n_L (\log \disc_L )^2)
\]
then
\[\tag{9.4}\label{(9.4)}
\psi_C(x) = \frac{|C|}{|G|}x - \frac{|C|}{|G|} \chi_0(g) \frac{x^{\beta_0}}{\beta_0} + R(x),
\]
where
\[
|R(x)| \leq x \exp(-c_{13} n_{L}^{-\frac{1}{2}} (\log x)^\frac{1}{2}),
\]
and where the second term on the right side of (\ref{(9.4)}) occurs only if $\zeta_L(s)$ has an exceptional zero $\beta_0$, and $\chi_0$ is the (real) character of $H = \mathrm{Gal}(L/E) = \langle g \rangle$ for which $\L(s, \chi_0, L/E)$ has $\beta_0$ as a zero.
\end{theorem}

\begin{proof}
If $\rho = \beta + i \gamma \neq \beta_0$ is a nontrivial zero of one of the $\L(s, \chi)$ with $|\gamma| \leq T$, then the unconditional bound of Lemma \ref{8.1} shows that 
\[
|x^\rho| = x^\beta \leq x \exp\Big(-c_{14} \frac{\log x}{\log \disc_L  T^{n_L}}\Big)
\]
for $x\geq 2$, $T \geq 2$.
Further, Lemma \ref{5.4} shows that
\[
\sum_\chi \sum_{\substack{\rho \\ |\rho| \geq \frac{1}{2} \\ |\gamma| \leq T}} \bigg| \frac{1}{\rho} \bigg| \ll \log T \log (\disc_L  T^{n_L}).
\]
Also,
\[
\sum_\chi \sum_{\substack{\rho \neq 1 - \beta_0 \\ |\rho| < \frac{1}{2}}} \bigg( \Big|\frac{x^\rho}{\rho}\Big| + \Big|\frac{1}{\rho}\Big|\bigg) \ll x^{\frac{1}{2}} \sum_\chi \sum_{\substack{\rho \neq 1 - \beta_0 \\ |\rho| < \frac{1}{2}}} \Big| \frac{1}{\rho}\Big| \ll x^\frac{1}{2} (\log \disc_L )^2,
\]
by Lemma \ref{5.4} and the fact that for $\rho \neq 1 - \beta_0$, $|\rho| \geq (4\log \disc_L )^{-1}$.
(If $\log \disc_L  = 0$, $L=\Q$, and the estimate holds trivially).
Finally,
\[
\frac{x^{1 - \beta_0}}{1 - \beta_0}  - \frac{1}{1 - \beta_0} = x^\sigma \log x \leq x^{\frac{1}{2}} \log x
\]
for some $\sigma$, $0 \leq \sigma \leq 1 - \beta_0$.
Collecting all these estimates gives us
\[\tag{9.5}\label{(9.5)}
    S(x, T) - \frac{|C|}{|G|} \chi_0(g) \frac{x^{\beta_0}}{\beta_0} 
     \ll \frac{|C|}{|G|} x \log T \log (\disc_L  T^{n_L}) \exp\bigg(- \frac{c_{14}\log x}{\log \disc_L T^{n_L}}\bigg) + \frac{|C|}{|G|} x^{\frac{1}{2}} (\log \disc_L )^2.
\]
We now choose 
\[\tag{9.6}\label{(9.6))}
T = \exp(n_{L}^{-\frac{1}{2}} (\log x)^{\frac{1}{2}} - \log \disc_L ).
\]
The estimate of the theorem then follows from (\ref{(9.5)}) and (\ref{(7.4)}).
\end{proof}

The deduction of Theorem \ref{1.1} and \ref{1.3} from the preceding theorem is now straightforward.
We first define the function
\[
\theta_C(x) = \sum_{\substack{\norm_{K/\Q}  \p \leq x \\ \p\text{ unramified} \\ \Scale[0.75]{\ArtinSymbol{L/K}{\p}} = C }} \log(\norm_{K/\Q}\p).
\]
Since there are at most $n_K$ ideals $\p^m$ ($\p$ prime) of a given norm in $K$,
\[\tag{9.7}\label{(9.7)}
\sum_{\substack{\p, m \\ m\ge 2 \\ \norm_{K/\Q} \p^m \le x}} \log(\norm_{K/\Q}\p) \ll n_K x^{\frac{1}{2}}
\]
by an elementary Chebyshev-type estimate.
This shows that the estimates of Theorems \ref{09-THM-9.1} and \ref{09-THM-9.2} hold when $\psi_C(x)$ is replaced by $\theta_C(x)$.
Theorems \ref{1.1} and \ref{1.3} now follow from these estimates for $\theta_C(x)$ by simple partial summation arguments.

We conclude this paper by indicating one way in which the GRH estimate  of Corollary \ref{1.2} can be slightly improved.
Instead of integrating
\[
\frac{1}{2\pi i} \frac{x^s}{s} F_C(x),
\]
we can integrate
\[
\frac{1}{2\pi i} \Bigl(\frac{y^{s-1} - x^{s-1}}{s - 1}\Bigr)^{2} F_C(x),
\]
where $y > x > 1$,
along the contour $B_{T, U}$ of Section \ref{6}.
We then first let $U \to \infty$, and then $T \to \infty$.
The integral from $\sigma_0 - i\infty$ to $\sigma_0 + i \infty$ gives us the term we are interested in, i.e.,
\[\tag{9.8}\label{(9.8)}
\sum_{\substack{\p \\\Scale[0.75]{\ArtinSymbol{L/K}{\p}}=C}} \frac{\log \norm \p}{\norm \p} r(\norm \p; y, x),
\]
where
\[
r(m;y, x) = \begin{cases} \log \frac{m}{x^2} & x^2 \leq m \leq xy, \\ \log \frac{y^2}{m} & xy \leq m \leq y^2, \\ 0 & \text{otherwise}, \end{cases}
\]
together with the contribution of the ramified primes and prime powers.
By Cauchy's theorem the value of the integral also equals the contribution of the poles of the integrand, which is
\[\tag{9.9}\label{(9.9)}
\frac{|C|}{|G|} \Bigl(\log \frac{y}{x}\Bigr)^2 - \frac{|C|}{|G|} \sum_{\chi} \bar{\chi}(g) \sum_{\rho} \Bigl(\frac{y^{\rho - 1} - x^{\rho - 1}}{\rho - 1}\Bigr)^2,
\]
where $\rho$ now runs through both the trivia and the nontrivial zeros of $L(s, \chi)$.
If we now choose $x = \log \disc_L $, $y = c_{14}x$, then for $c_{14}$ sufficiently large (and on the assumption of GRH) the main term in (\ref{(9.9)}) will dominate both the sum over the zeroes and of the ramified prime and prime power factors, so that (\ref{(9.8)}) will have to be nonzero.
Hence there will be a prime $\p$ with $\ArtinSymbol{L/K}{\p} = C$ and
\[
\norm \p \leq y^2 \leq c_{14}^2 \log^2 \disc_L .
\]

% \tag{5.10}\label{(5.10)
\end{document}